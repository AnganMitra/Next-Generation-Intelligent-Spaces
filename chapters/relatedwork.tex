\section{Related Work}
\label{sec: relatedworks}


Historically, buildings were not designed to cater to forms of ambient intelligence rather optimized spatially for acceptable levels of thermal comfort, indoor ventilation, and privacy.
With the passage of time, building became a composite of observable and controllable elements.


% Cheap availability of sensors has spurred ad-hoc incorporation into zones of a buildings. 
% It is not clear from the literature which spaces need sensors and what are the zones one can save up on installation cost. 
% The context stems from studying the gaps in smart building literature and industry focusing on the three major pain points of such technology.

% This essence of distributed computing is naturally evolves from  embedded systems scattered across spaces in such a building. 


\subsection{Review of Technology Acceptance}


 
 Smart applications \cite{wong2005intelligent} for buildings has been developed mainly for monitoring, analysis, and control of thermal units like Heating Ventilation Air Conditioning (HVAC) units, illumination channels etc.
 A 2019 review \cite{jia2019adopting} of the smart building industry states the major pain points towards technological adaption.
 High installation costs, obscurity on data storage policies, and privacy concerns impede the acceptance \cite{hojjati2016evaluation} of Internet of Things in buildings.
Typically a smart building applications thrive on real time sensor data for monitoring or actuation.
Research shows that analysing sensor streams can reveal sensitive patterns about occupancy \cite{garg2000smart} or usage.
Consequently, privacy becomes a major concern for occupants in a building due to a non zero possibility of data leak. 
The cost  of constructing \cite{ma2017business} a smart building is usually 1.2-1.8 times a non-smart counterpart.
This initial capital poses the second barrier for a stake holder \cite{xu2019platform} to overcome before system installation.
But before a technical deployment \cite{ma2016market}, the smart solution needs go through a pre-evaluation stage before finalizing the bill of materials.

We add to the literature a methodology for pre-integration, and initial planning with a market segmentation analysis. 
The motivation for a minimalist sensor design aids to non-intrusive sensing that can either be ad-hocly assembled or composed with industrial grade sensors.
The proposed system optimises logical grouping that encodes a human-interpretable data sharing policy.

% within the clarity on the data sharing policy, we encode t
% Not everywhere there may be a pressing need to install industrial grade sensors  a mix of Do It Yourself 
% with a focus to lower the capital and operating cost of a generic building solution. 
% The solution gives a business alternative to the classical fix and forget strategy of distributing sensors.
% We perform a market segmentation of sensing to 




\subsection{Ambient Intelligence}

Multiple cyber physical systems like sensors and actuators work in cohesion to maintain a desired quality of ambience and indoor comfort of a building.
Some examples of non-intrusive ambient sensors are temperature, humidity, luminosity.
Data values recorded by a type of sensor is usually dissimilar across different buildings or between separate zones in a building.
Empirical Mode Decomposition (EMD) \cite{fontugne2012empirical} of a continuous variable such as temperature, humidity or luminosity yields Intrinsic Mode Functions (IMF)\cite{ayenu2010criterion}.
This has been shown useful for structural health monitoring  \cite{barbosh2020empirical} for buildings.
K means clustering over the space of IMF for all the sensors is shown to be effective \cite{hong2013towards} in identifying non identical sensors.
This approach is further extended \cite{YOGANATHAN20181206}  by using information loss to eliminate weak candidate points from a cluster to obtain a solution to the Optimal Sensor Location Problem (OSLP).
A key experimental insight of OSLP  is that there can be more than one group, which are not worse from one another or Pareto Optimal \cite{censor1977pareto} in nature.

We broaden the scope of the OSLP to find the minimal support sensors that can predict the redundant sensor values. 
The aim here is to generate a virtual sensor field for a building that is spatially optimised to have the least generalization error in prediction. 
For example, the behaviour of a group of temperature sensors situated across multiple zones probably can be learnt by an optimal fraction of embedded devices. 
Noticeably the energy footprint in powering up all sensors is higher than a fraction of the same.
% This opens up the scope for a dynamic optimization where primal objectives should be on lowering the capital and operating cost, controllable enhancement of privacy by design while maintaining a fair level of accuracy
% Although it is clear from the literature about existing techniques to find optimal locations, to create a generalized 
%  The optimal sensor placement problem  uses this solution to identify significant locations in a building to place sensors. 

% We define a virtual field as an approximation of temperature, humidity and luminosity sensors and control channels the illumination power and ambient power.
% The target of such systems is to maintain indoor comfort which is generally modelled as the net deviation from desired set points of the control channels.

