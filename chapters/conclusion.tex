\section{Conclusion}
\label{chap:conclusion}

 
This work emphasized the role of spatio-temporal knowledge to bring down the operating cost of building management system with an added layer of privacy.
Generating a distributed virtual field contributes to the problem of monitoring spaces non intrusively. 
The virtual sensor field can be utilized to approximate the original data in case of sensor fault or turn off unnecessary permissible sensors.
We observe that the missing sensor approximation can be kept competitively accurate with bidirectional power-ambience converters with explainable insights.

The extension of the work can be studying the Utopian sensor placement across zones with theoretical learning guarantees.  
It is interesting to evaluate the performance in an online learning setting which can be the next step towards auto updating spatio temporal models.

